\documentclass[%
        paper=a4,
	% BCOR=10mm, % Binding CORrection
	oneside,			% two sided printing
        %cleardoublepage=empty,         % empty backsides for two sided printing
        twocolumn=false,
	titlepage=false,			% support for title environment
        abstract,
	numbers=noenddot,		% numbers (in headlines) without ending points
	headsepline,			% line below header
	%footsepline,			% line above footer
	headings=normal,                % size of header is normal
	fontsize=11pt,			% fontsize in points
	pagesize=pdftex,		% define typearea for pdftex usage
        headheight=46.96317pt,          % value based on MyThOS logo
        parskip=off                     % or half*
]{scrartcl}

% ------------------------------------------------------------------------------
% character encoding and fonts for the document
\usepackage[T1]{fontenc} % should be loaded before inputenc
\usepackage[utf8]{inputenc}	    % unicode, newer than utf8x
\usepackage{ae}
\usepackage{aecompl}
\usepackage[T1]{url}			% web adresses with t1 style
\urlstyle{tt}					% web adresses in tt-style
%\usepackage[UKenglish]{babel}
\usepackage[ngerman,UKenglish]{babel}

% ------------------------------------------------------------------------------
% font setup
\usepackage{microtype}
\usepackage{ellipsis}			% corrects wrong spaces after dots
\usepackage{amsmath}
%\usepackage{amsfonts}
\usepackage{amssymb}
\usepackage{latexsym}
\usepackage{dsfont}
\usepackage{textcomp}

%% bera mit euler
\usepackage{bera,amssymb,eufrak}
\usepackage[euler-digits,euler-hat-accent]{eulervm}

\setlength{\parskip}{0.5em plus 1em minus 0.5em}
\clubpenalty 2000
\widowpenalty 2000
%\setlength{\tolerance}{2000}
\setlength{\emergencystretch}{20pt}   % kann man evtl. auf 20 erhöhen
%\setlength{\hfuzz}{1pt}
\setlength{\hfuzz}{10pt}

% ------------------------------------------------------------------------------
% pdf creation
\usepackage[%
  pdftex,
  colorlinks=true,
  urlcolor=red,
  citecolor=red,
  linkcolor=red,
  bookmarks=true,
  breaklinks=true
]{hyperref}

% ------------------------------------------------------------------------------
% graphics
\usepackage[usenames,dvipsnames]{xcolor}
\usepackage{graphicx}
\usepackage{subfigure}
\usepackage{tikz}
\usetikzlibrary{calc,trees,positioning,arrows,chains,shapes.geometric,%
    decorations.pathreplacing,decorations.pathmorphing,shapes,%
    matrix,shapes.symbols}
\usepackage{pgfplots}
\pgfplotsset{compat=newest}
\usetikzlibrary{pgfplots.statistics} % For Box plots
\usepgfplotslibrary{units} % Allows to enter the units nicely


% ------------------------------------------------------------------------------
% listings
\usepackage{listings}		% include listings
\newcommand{\code}{\lstinline}
\lstloadlanguages{C++}
\lstset{
  basicstyle=\ttfamily\footnotesize,
  keywordstyle=\bfseries,
  stringstyle=\textit,
  emphstyle=\textit,
  commentstyle=\textit,
  breaklines=true,
  breakatwhitespace=true,
  breakautoindent=true,
  morekeywords={},
  numbers=left,
  numberstyle=\tiny\color{black},
  stepnumber=1,
  numbersep=5pt,
  tabsize=2,
  showspaces=false,
  showstringspaces=false,
    xleftmargin=17pt, 			% left margin where linenumbers and frames do start
    framexleftmargin=17pt, 		% left margin of frame (which will be colored)
    framesep=0pt,
    belowcaptionskip=0pt,
    captionpos=b,
    framerule=1pt
}

\usepackage{scrhack}
\usepackage{float}
\floatstyle{plain}
\newfloat{listing}{tbhp}{myc}
\floatname{listing}{Listing}

% ------------------------------------------------------------------------------
% tables
\usepackage{multirow}
%\usepackage{ragged2e}
\usepackage{tabularx}
\usepackage{longtable}
\usepackage{booktabs}   % toprule, midrule in Tabellen
\newcolumntype{Y}{>{\raggedright\arraybackslash}X}
\usepackage{caption}

\usepackage{collcell}
\newcolumntype{L}{>{\collectcell\lstinline}l<{\endcollectcell}}
\newcolumntype{R}{>{\collectcell\lstinline}r<{\endcollectcell}}

\usepackage{bytefield}

%------------------------------------------------------------------------------
\usepackage{xspace}
\newcommand{\mythos}{\textsc{MyThOS}\xspace}
\def\cc{{C\nolinebreak[4]\hspace{-.05em}\raisebox{.4ex}{\tiny\bf ++}}}

\usepackage[textsize=tiny]{todonotes}

%\usepackage{titling}
\usepackage{enumerate}
\usepackage{epigraph}
\renewcommand{\epigraphsize}{\footnotesize}
\setlength{\epigraphwidth}{0.618\textwidth}
\usepackage{placeins}

\titlehead{
\begin{center}
  \noindent\rule{1.0\textwidth}{1pt}\\[1cm]
  \includegraphics[scale=0.5]{MyThOS_Logo}\\[0.5cm]
  \sffamily
  \small
  \begin{tabularx}{\textwidth}{lX}
    Förderkennzeichen: & 01IH130003 \\
    Vorhabensbezeichnung: & MyThOS \\
    & Modulares Betriebssystem für Massiv Parallele Anwendungen
  \end{tabularx}\\[0.5cm]
  \noindent\rule{1.0\textwidth}{1pt}
\end{center}
%\vspace{1cm}
}


\usepackage{scrlayer-scrpage}
\makeatletter
\title{\@myTitle}
\renewcommand*{\title}[1]{\def\@myTitle{#1}}
\setkomafont{pageheadfoot}{\sffamily\small}
\lohead{%
\@myTitle\\%
\headmark%
}
\cohead{}
\rohead{\includegraphics[height=1.5cm]{MyThOS_Logo}}
\makeatother
\pagestyle{scrheadings}

%\usepackage{showframe}

% ------------------------------------------------------------------------------
% layout
\usepackage{multicol}
\usepackage{marginnote}
\renewcommand*{\marginfont}{\footnotesize}
\typearea[current]{last}		% recalculate type area

% units and numbers
\usepackage{siunitx}
\sisetup{scientific-notation = true}%
\sisetup{
  round-mode          = places,
  round-precision     = 2,
}

% ------------------------------------------------------------------------------
% funding notice on title page
\newcommand\blfootnote[1]{%
  \begingroup
  \renewcommand\thefootnote{}\footnote{#1}%
  \addtocounter{footnote}{-1}%
  \endgroup
}

\makeatletter
\g@addto@macro{\maketitle}{\blfootnote{Das diesem Bericht zugrunde liegende Vorhaben wurde mit
  Mitteln des Bundesministerium für Bildung und Forschung unter dem
  Förderkennzeichen 01IH13003 gefördert. Die Verantwortung für den
  Inhalt dieser Veröffentlichung liegt bei den Autoren.}}
\makeatother


\title{MyThOS D3.3 Softwareentwicklungsplan}
\author{Stefan Bonfert, Vladimir Nikolov, Robert Kuban, Randolf Rotta}

\hypersetup{
  pdftitle={MyThOS D3.3 Softwarentwicklungsplan},
  pdfsubject={MyThOS Deliverable},
  pdfauthor={Stefan Bonfert, Vladimir Nikolov, Robert Kuban, Randolf Rotta},
  pdfkeywords={MyThOS} % comma separated
}

\begin{document}
\selectlanguage{ngerman}
\maketitle

\begin{abstract}

Dieses Dokument umfasst den Strategieplan für die Weiterentwicklung des
Prototypen und Anleitungen für Systementwickler.
% Abschnitt~\ref{sec:kernel-objects} gibt Beispiele und Hinweise für die
% Entwicklung eigener Kernel-Objekte, die Behandlung von Systemaufrufen, die
% Synchronisation asynchroner Aufgaben, sowie die Verwaltung von Beziehungen
% zwischen Kernel-Objekten.
Es fasst somit nützliche und geplante Erweiterungen des bestehenden
Kernels zusammen. Dazu gehören zum Beispiel Verbesserungen bei der Fehler- und
Performanceanalyse, Schnittstellen zur Konfiguration der Interruptbehandlung und
Capability Transfers zwischen Kommunikationsportalen.
 
\end{abstract}

\newpage
\tableofcontents

% --- content ------------------------------------------------------------------
%\section{Übersicht}


% --- content ------------------------------------------------------------------
\selectlanguage{UKenglish}
% \section{Kernel Development}
% \label{sec:kernel-objects}
% 
% This section describes the steps necessary to create a new kernel
% object. First, the \texttt{IKernelObject} interface
% (Listing~\ref{lst:ikernelobject}) is discussed. It has to be
% implemented by all kernel objects in order to participate in the
% resource management and processing of system calls.  Then, the
% allocation and initialisation of kernel objects through factories is
% described.  Finally, an overview over the synchronisation monitors and
% the weak references mechanism is given.
% 
% The kernel source includes an example object, which serves as
% a starting point for the creation of custom Kernel Objects and as a
% mock for testing object allocation and deletion. For further
% information, consider the documentation embedded into the source
% code. Most of the code embedded here is part of the Example object.
% 
% \begin{lstlisting}[float, label=lst:ikernelobject, caption=The \texttt{IKernelObject} interface.]
% class IKernelObject : public ICastable {
% public:
%   virtual Range<uintptr_t> addressRange(Cap self);
%   virtual optional<Cap> mint(Cap self, CapRequest request);
%   virtual optional<void> deleteCap(Cap self, IDeleter& del) = 0;
%   virtual void deleteObject(Tasklet* t, IResult<void>* r) = 0;
%   virtual Range<uintptr_t> objectRange() const = 0;
%   virtual optional<CapEntryRef> lookup(Cap self, CapPtr needle, CapPtrDepth maxDepth);
%   virtual void invoke(Tasklet* t, Cap self, IInvocation* msg);
% };
% 
% class ICastable {
% public:
%   virtual optional<void const*> vcast(TypeId id) const;
% };
% \end{lstlisting}
% 
% \subsection{Ressource management}
% 
% The \texttt{addressRange()} method is used by the resource inheritance
% tree for parent-ship tests. The method has to return the physical
% address range that is used by the object. The range has to be a
% superset of the address ranges of all derived child objects in the
% capability inheritance tree. Kernel objects that will have no children
% can simply use the default implementation.
% 
% \texttt{deleteCap()} and \texttt{deleteObject()} are both part of the
% multistep deletion process. \texttt{deleteCap()} notifies the object
% that a capability is going to be deleted. For most objects, this will
% have no effect for reference and derived capabilities. For original
% capabilities, the objects recursively deletes its capability entry and
% schedules itself for asynchronous deletion using the \texttt{IDeleter}
% object given as a parameter. Listing~\ref{lst:deletecap} shows a simple example
% for an object without any embedded capability entries.
% 
% \begin{lstlisting}[float, label=lst:deletecap, caption=Example handler for capability revocation.]
% optional<void> ExampleObj::deleteCap(Cap self, IDeleter& del) {
%   if (self.isOriginal()) del.deleteObject(del_handle);
%   return Error::SUCCESS;
% }
% \end{lstlisting}
% 
% \texttt{deleteObject()} carries out the final step of the asynchronous
% deletion. The object will use its deletion monitor to wait for all
% outstanding asynchronous request to complete and then asynchronously
% requests the deletion from the \texttt{KernelMemory} object it was
% allocated from. An example is shown in
% Listing~\ref{lst:deleteobject}. The memory's \texttt{release()} method
% will call the object's \texttt{objectRange()} to query the address
% range that is used by the object itself. First, the objects virtual
% destructor \texttt{~IKernelObject()} is called in order to let the
% object free all contained additional memory blocks. Then, the objects
% memory is freed. Finally, the deleter is notified about the completion
% by replying to the \texttt{IResult} pointer.
% 
% \begin{lstlisting}[float, label=lst:deleteobject, caption=Example handler for asynchronous object deletion.]
% void ExampleObj::deleteObject(Tasklet* t, IResult<void>* r) {
%   monitor.doDelete(t, [=](Tasklet* t){ this->_mem->release(t, r, this); });
% }
% \end{lstlisting}
% 
% The \texttt{mint()} method is used when creating a reference or derived capability. This mechanism allows to restrict access rights, add communication badges and similar by altering the data portion of the capability according to the request argument. The meaning of the request arguments is defined by the actual object because the capability data portion is interpreted only by the object itself. If a kernel object does not accept the minting request, it should return an error code.
% 
% \subsection{System Calls as Capability Invocation}
% 
% The \texttt{lookup()} method is used by the system call entry code in
% order to find the target object that matches the given capability
% pointer. The search in the caller's capability space recursively walks
% through its capability maps. The default implementation of this method
% fails and returns an error code. A custom implementation is only
% necessary if the object acts as a capability map.
% 
% The \texttt{invoke()} method implements the receiving side of the
% capability invocation to a kernel object. Its default implementation
% simply returns with an error. Listing~\ref{lst:invokehandler} shows an
% example of the invocation handling. The \texttt{invoke()} method
% dispatches the message based on the message label and asynchronously
% calls the appropriate implementation. The implementation checks the
% caller's access rights by inspecting the capability \texttt{self} that
% was used to access the object. Then, the arguments are copied from the
% message into local buffers and checked for validity. The copy is
% crucial for security because otherwise concurrently running
% application threads could manipulate the arguments after the sanity
% check have been passed. After carrying out the request, the object
% replies and releases the its monitor.
% 
% \begin{lstlisting}[float, label=lst:invokehandler, caption=Example invocation 
% handler.]
% void ExampleObj::invoke(Tasklet* t, Cap self, IInvocation* msg) {
%   switch (msg->getLabel()) {
%   case 0:
%     return monitor.request(t, [=](Tasklet* t) { printMessage(t, self, msg); });
%   default:
%     msg->replyResponse(Error::INVALID_REQUEST);
%   }
% }
% 
% void ExampleObj::printMessage(Tasklet* t, Cap self, IInvocation* msg) {
%   // may use data portion of self for access control and badges...
%   auto msgdata = msg->getMessage();
%   // do something useful with the message data...
%   message.replyResponse(Error::SUCCESS);
%   monitor.requestDone();
% }
% \end{lstlisting}
% 
% The \texttt{IInvocation} interface provides several more methods that
% help, for example, to look up additional kernel objects in the callers
% capability space, access the callers \texttt{ExecutionContext}, and
% the caller's logical address space configuration.
% 
% \subsection{Interactions between Kernel Objects}
% 
% When a capability lookup is used to retrieve a pointer to another
% kernel object, this pointer is still typed as
% \texttt{IKernelObject}. This is sufficient to pass invocation messages
% to the object. However, the invocation mechanism is complex and
% possibly costly. Therefore it would be more efficient and convenient
% to call the kernel object's implementation-specific methods
% directly.
% 
% Normal C++ programs use the compiler-generated runtime type
% information (RTTI) and the \texttt{dynamic\_cast} conversion. This
% information is not available in the kernel in order to reduce its
% size. Instead, the conversion is accomplished through the
% \texttt{ICastable} mechanism. The method \texttt{vcast()} is
% implemented by each kernel object in order to hand out pointers to
% more specific interfaces. Listing~\ref{lst:vcast} shows an example.
% The converted pointer is then retrieved via the \texttt{cast()} helper
% method as shown in Listing~\ref{lst:cast}.
% 
% \begin{lstlisting}[float, label=lst:vcast, caption=Example run-time type conversion in the \texttt{KernelMemory} object.]
% virtual optional<void const*> vcast(TypeId id) const override {
%   if (id == TypeId::id<KernelMemory>()) return this;
%   if (id == TypeId::id<IAllocator>()) return static_cast<IAllocator const*>(this);
%   return Error::TYPE_MISMATCH;
% }
% \end{lstlisting}
% 
% \begin{lstlisting}[float, label=lst:cast, caption=Casting a kernel object into a more specific type.]
% auto alloc = ptr->cast<IAllocator>();
% if (alloc) { // checks whether the conversion was successful
%   alloc->allocate(...);
% }
% \end{lstlisting}
% 
% 
% 
% \subsection{Object Creation}
% 
% Kernel objects are created with memory from a single
% \texttt{KernelMemory} via the help of a factory object.  The only
% exception are a few initial objects that exist statically.  The
% factory has to implement the \texttt{IFactory} interface, which is
% shown in Listing~\ref{lst:ifactory}. 
% 
% \begin{lstlisting}[float, label=lst:ifactory, caption=The \texttt{IFactory} interface.]
% class IFactory {
% public:
%   virtual optional<void> factory(Cap umcap, IAllocator* mem, IInvocation* msg,
%                                  CapEntry* tgtentry) = 0;
% };
% \end{lstlisting}
% 
% The \texttt{umcap} capability belongs to the \texttt{KernelMemory}
% that called the factory.  This value is later needed to complete the
% insertion into the resource tree.  The \texttt{mem} argument points to
% the \texttt{IAllocator} interface of the \texttt{KernelMemory}.
% Because the memory calls the factory the memory's synchronisation
% monitor has been entered, synchronous calls to \texttt{mem} can be
% used safely.  The created object has to store \texttt{mem} in order to
% be able to release its memory later.  The \texttt{tgtentry} points to
% the capability entry that shall receive the first capability that
% points to the newly created kernel object.  Finally, the \texttt{msg}
% arguments contains the invocation buffer that was passed from userland
% with additional initialisation parameters for the new object.
% 
% An example factory implementation is shown in
% Listing~\ref{lst:factory}.  First, the target capability entry is
% acquired and locked in order to prevent data races (line 3).  Then,
% memory for the object is allocated and the object is constructed in
% this memory.  The factory and the object constructor can allocate
% additional memory from \texttt{mem} (line 2).  If the allocation
% fails, the memory has to be released and the target capability entry
% has to be reset (lines 5--7).
% 
% Finally, the original capability of the created object is inserted as
% child of the \texttt{KernelMemory} that it was allocated from (line
% 10).  The parent entry is retrieved from the invocation message and
% should match \texttt{umcap}.  The operation fails when the parent
% object was deleted concurrently.  In this case, the object and the
% target entry have to be released (lines 11--14).  The original
% capability can be either stored as member variable in the created
% object, as shown in the example, or can be stored directly in the
% target entry.  In the example, a reference capability is written to the
% target entry.
% 
% \begin{lstlisting}[float, label=lst:factory, caption=An example factory implementation.]
% optional<void> ExampleFactory::factory(Cap umcap, IAllocator* mem,
%                                        IInvocation* msg, CapEntry* tgtentry) {
%   if (!tgtentry->acquire()) return Error::LOST_RACE;
%   auto obj = mem->create<ExampleObj,64>(mem, ...); // create with 64b alignment
%   if (!obj) {
%     tgtentry->reset();
%     return Error::INSUFFICIENT_RESOURCES;
%   }
%   Cap tgtcap(obj);
%   auto res = cap::inherit(msg->getCapEntry(), obj->ownRoot, umcap, tgtcap);
%   if (!res) {
%     mem->release(obj);
%     tgtentry->reset();
%     return res.state();
%   }
%   return cap::inherit(obj->ownRoot, tgtentry, tgtcap, tgtcap.asReference());
% }
% \end{lstlisting}
% 
% 
% 
% \subsection{Serialisation Monitors}
% 
% The monitors are used by kernel objects to serialise asynchronous
% method calls and select the place (hardware thread) that processes the
% calls.  Currently, four monitor variants are available.  The choice
% depends on the synchronisation needs.  In most cases, the
% \texttt{NestedMonitorDelegating} is a good choice.
% 
% \paragraph{DeletionMonitor.}
% This monitor implements a reference counter that is used to delay the
% deletion request until all counted references were released.  It is
% used as base class for the other monitors.  The methods
% \texttt{acquireRef()} and \texttt{releaseRef()} increase or decrease
% the reference counter, respectively.  The \texttt{doDelete()} method
% schedules a Tasklet to be executed the next time the reference counter
% reaches zero.  The Tasklet is processed at the place of the
% \texttt{releaseRef()} caller.  If the reference counter is zero
% already, the Tasklet processed immediately.
% 
% \paragraph{SimpleMonitorHome.}
% This variant comes close to the classic monitor concept.  The
% \texttt{request()} method schedules the given Tasklet at a predefined
% place called \texttt{home}.  At the end of each request handler
% implementation, \texttt{requestDone()} has to be called in order to
% release the reference counter.  Because all Tasklet scheduled by the
% monitor are processed on the same hardware thread, they all are
% mutually exclusive.
% 
% \paragraph{NestedMonitorHome.}
% This extension of the previous monitor differentiates between
% asynchronous \emph{requests} and \emph{responses} in order to
% implement a form of nested locking.  The next requests is processed
% only after the previous has finished by calling
% \texttt{requestDone()}.  Requests can issue sub-requests to other
% kernel objects and the last response has to call
% \texttt{responseAndRequestDone()} instead of just
% \texttt{responseDone()}.  Developers should be careful with cyclic
% dependencies between kernel objects because the mutually exclusive
% request are prone to deadlocks -- like in any scenario with nested
% locks that can be entered just once.
% 
% \paragraph{NestedMonitorDelegating.}
% This monitor has the same \emph{request}/\emph{response} interface as \texttt{NestedMonitorHome} but
% the monitor is not bound to a specific home place.  Instead, the first
% request that acquires exclusive access sets the home to its caller's
% place.  All responses to sub-requests will processed at this
% place.  Further requests that do not acquire exclusive access will be
% processed at this place, too.  The last \texttt{requestDone()} or
% \texttt{responseAndRequestDone()} releases the exclusive access, such
% that later requests select a new home.  This behaviour is based on the
% concept of \emph{delegation locks}.
% 
% 
% \subsection{Connecting Objects through Weak Reference Capabilities}
% 
% Pointers between kernel objects require special care because the
% target object can be deleted concurrently.  This would leave dangling
% pointers, which result in fancy bugs and headaches.  In many cases,
% the associations between kernel objects are established by the
% transfer of access rights.  In \mythos this is implemented by passing
% a capability to the kernel object and the capability revocation can be
% used to clean up such pointers.  This idea is implemented by the
% \texttt{CapRef} class.
% 
% \begin{lstlisting}[float, label=lst:capref, caption=The weak reference capability interface.]
% template<class Subject, class Object, class Revoker=RevokeByUnbind>
% class CapRef : public CapRefBase {
% public:
%   optional<void> set(Subject* subject, CapEntry& src, Cap srcCap);
%   optional<Object*> get() const;
%   void reset();
% protected:
%   virtual void revoked(Cap self, Cap orig);
% };   
% \end{lstlisting}
% 
% Listing~\ref{lst:capref} summarises the principal interface of
% \texttt{CapRef}. Basically, the object contains a capability entry
% that holds a reference capability and a pointer to the
% \texttt{subject} capability holder.  The \texttt{set()} method
% inherits the reference from the \texttt{src} and \texttt{srcCap} pair.
% It fails if a concurrent call to \texttt{set()} was faster or when the
% source entry was deleted concurrently.  The \texttt{get()} method
% returns a pointer to the capability's object casted to the desired
% object interface.  It fails if the capability was never set or was
% revoked meanwhile.  The \texttt{reset()} method clears the reference
% and called automatically from \texttt{set()}.
% 
% During capability revocation, the \texttt{revoked()} method will be
% called synchronously.  It can be overridden by derived types.  The
% default implementation calls 
% \lstinline{Revoker::apply(subject, object, self, orig)}.
% The default \texttt{RevokeByUnbind} calls
% \lstinline{subject->unbind(object)} synchronously in order to notify the reference
% holder. An example usage is given in Listing~\ref{lst:caprefex}.
% 
% \begin{lstlisting}[float, label=lst:caprefex, caption=An example use of weak references.]
% class Foo : public IKernelObject {
% public:
%   optional<void> bindPortal(CapEntry& pcap); // calls portal.set(...)
% protected:
%   CapRef<Foo, IPortal> portal;
%   friend class RevokeByUnbind;
%   void unbind(optional<IPortal*> o); // reacts to the revocation
% };
% \end{lstlisting}
% 

\section{Introduction}
\label{sec:intro}

This document recaps a strategy plan for the further development of the MyThOS
operating system and its prototype. It summarizes a bunch of useful and
nice-to-have improvements, like for example for error and performance analysis, interfaces
for the configuration of the interrupt handling and capability transfers
between communication portals. It also outlines the development strategy of the
partners in terms of future academic activities and projects, that will help to
sustainably improve the system.

In the second phase of the project the MyThOS system and its architecture were
completely redesigned and re-implemented from scratch. The lessons learned from
the previous version helped us to streamline the kernel into a more consistent
shape and operation. Logical and functional gaps are not pushed up anymore to
the responsibility of the application developers, but were filled and closed
with clear definitions, requirements and architectural elements of the system.
In the course of that development various nice-to-have functionalities were left
aside or postponed, as the focus was initially set on the development of the new
version of the base kernel. Such functionalities and features are presented and
discussed later in Section~\ref{sec:extensions}.

MyThOS is planned to serve as a basis for various further research projects and
initiatives. Thereby, aspects like different scheduling policies, task models,
resource management strategies (CPU, energy, network), fault tolerance and
support for time-critical applications will be investigated. The fields of
application and research range from heterogeneous and embedded systems, over
large scale infrastructures to time-critical and real-time applications and
systems. Concrete developments and projects as well as cooperations with
industrial and academic partners will be announced and referenced on MyThOS's
website (\url{https://manythreads.github.io/mythos/}).

Furthermore, we aim to extend the code-base with further (example) applications
and to port the system to different hardware architectures (e.g. ARM, PPC,
particular Microcontrollers, etc). Also modules for physical storage, network
communication, graphical display and device drivers for further peripherial
hardware are still on the roadmap. 

Besides this, theoretical and practical aspects of MyThOS are already and will
be further integrated in various lectures, seminars, student thesis and labs at
the Ulm University and at the BTU Cottbus. 

MyThOS was also published on GitHub with the aim to build an open source
developer community and to provide a well-known platform for further
developments and improvements.

In the following Section~\ref{sec:extensions} we now summarize some of the
directions for technical improvements of the kernel.

\section{Strategy for further Development}
\label{sec:extensions}

\subsection{Capability Transfer}

Capability transfer and capability unwrapping in inter-process calls
are not necessary for the applications targeted by \mythos.  However,
they promise to simplify the interaction between unrelated application
without a common supervisor.

In order to support the future implementation of such transfers, the
invocation buffer format mirrors the seL4 message format and reserves
the necessary space.  In order to support a capability transfer, a
portal implementation would read the desired target capability address
from the receiver's invocation buffer, look up the respective entry in
the receiver's capability space, and insert a reference capability
there.

\subsection{Image-based Debugging Facilities}

Because the system memory in \mythos is just 4GiB large, it is
feasible to create an image of the whole system memory by reading it
from the host over PCIe.  Host access to the coprocessor memory is
cache coherent if the host-side caching is disabled. Although it is
possible create a kernel image without further kernel support,
concurrent changes to the kernel memory can easily lead to an
inconsistent image and interesting details about the state of each
hardware thread are hidden in their registers.

Two approaches can be implemented in order to obtain a consistent
kernel image: \emph{Snapshot Images} capture a consistence state by
issuing a non-maskable interrupt (NMI), which dumps the processor
state into each thread's NMI stack and, and then, busy waits in the
interrupt handler until the image is captured.  In contrast, a
\emph{Checkpoint Image} is created by using a interrupt to force all
threads into the kernel, but instead of waiting in the NMI handler,
all threads wait before the next Tasklet is executed. This allows to
create a kernel image that is easier to analyse because all
transactions and critical sections were completed.

\subsubsection{Doorbell Interrupts and User-Level Interrupt Handling}
The Intel Xeon Phi Processor has a doorbell interrupt, which can be
triggered from the host software in order to wakeup the sleeping
processor or interrupt its current activity in order to handle urgent
requests. This mechanism is based on configuration registers in the
processor's PCIe MMIO space and the XeonPhy's IOMMU.  The respective
configuration can be added through a KNC-specific source code
module. One imminent application would be the interrupt broadcast for
debugging purposes.

The present interrupt handling in \mythos is sufficient to handle
traps and interrupts inside the kernel but is not extensible. A useful
improvement would be the implementation of user-level interrupt
handling. The basic idea is to represent interrupt gates as kernel
objects and triggered interrupts send a message or notification to a
portal or execution context.

\subsection{Exploiting Hardware Transactional Memory}
% was ist das / wozu?
Hardware Transactional Memory such as Intel's TSX extension can speed
up the execution by enabling lock elision and much simpler
non-blocking algorithms.  Previous related work about transactional
memory in
microkernels~\cite{smejkal2015transactional,fuchs2014hardware} focused
on the latency of the IPC path.  The results were disappointing
because one or two atomic exchange or CAS operations are difficult to
beat.

% status in mythos?
% wie könnte es besser unterstützt werden?
\mythos contains quite a lot non-blocking algorithms, especially for
double-linked queues of pending messages and the resource inheritance
tree. These can be replaced by hardware transactional memory in order
to reduce complexity and busy waiting. Very likely, TSX would not
speed up the serialisation monitors and the tasklet queue.

\subsection{Conservative Sleeping}

When there is no work to do, a hardware thread should switch into a
sleep state in order to minimize energy consumption and to lend its
thermal budget to other cores.  Currently, a hardware thread in
\mythos starts sleeping as soon as there are no more executable
Tasklets and Execution Contexts (application threads).  However, this
is not optimal because waking up from a sleep state requires an
interrupt to be send by another thread, which imposes a latency
penalty. Entering and exiting deeper sleep modes also adds a
considerable overhead.

A possible solution to this problem is introducing an active waiting
phase, which resembles the \emph{pause} phase for mutexes first
introduced by Ousterhood.  Heuristics might be used to find a tradeoff
between active waiting phase length, hence energy consumption, and the
latency penalty of waking up from a sleep state. \mythos can support
the development of such heuristics by providing diagnostic data like
active waiting time or sleeping phase length.

\subsection{Sleeping through MWAIT}

Many x86 Prozessors, except the Intel XeonPhi Knight Corner, support a more efficient sleep\&wakeup mechanism for the hardware threads. It is based on monitoring a cache line for access by other hardware threads with the \texttt{MONITOR} instruction. Then the thread can sleep with \texttt{MWAIT} and is woken up whenever the monitored line is modified or evicted from the thread's cache. 

The benefit of this approach is, that client threads that send a Tasklet do not need to check whether they have to send an IPI interrupt. The sleeping thread does not need to go through the interrupt entry routine when waking up. In \mythos this would allow to leave out the release of the thread's Tasklet queue prior to sleeping.

\subsection{Adapt Tracing for Tasklet Queues and Synchronisation Monitors}
The previous implementation of \mythos used fixed-size Ringbuffers for the exchange of Tasklets between hardware threads. In addition, Tasklets were allowed to enter multiple queues at the same time and were allocated dynamically. This introduced interesting challenges for the global reconstruction of thread-local traces. 

In contrast, the present design is much simpler because all Tasklets have to follow a request-response cycle between kernel objects. This allows to simplify the trace recording and reconstruction. The port of the trace subsystem to the new design is still pending.


\bibliographystyle{alpha}
\bibliography{literature}

\end{document}
